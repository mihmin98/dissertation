\chapter{Introduction}

The video games industry is one that has been around since 1970 and over the years it became one of the most successful forms of entertainment. As the years went by, the complexity of the games also increased, requiring more time to develop and more manpower. In the early days, mainstream titles would have smaller teams, usually comprised of 10 to 20 people, however, as time went by, so did the teams increase in size, with studios having over hundreds of people working on a single game.

One important aspect of video games is the AI. It is a tough challenge for a developer to create and implement an AI to perform actions in such a way that feels \emph{natural} and not scripted, so that the player's immersion is not broken. As the video games evolved, so did the AI, becoming more complex and having more ways to interact with the player and the environment.

A solution for this problem would be to implement Reinforcement Learning techniques to train the AI to perform the required tasks. Some advantages of this approach would be that the developer would need to create a set of rules for the AI, and then reward or punish the it for its taken actions. Another advantage would be that it would be simple to implement this behaviour, if the actions that the AI perform are similar to the actions that the player performs since they have already been implemented. However, this approach also has some negative aspects. For starters, this training could take a long period of time, depending on the complexity of this problem and the allotted resources for the training process. Another disadvantage would be that the even if the developers do not have to manually implement the AI logic, they still haev to create punishments and rewards for the training agent, and also have to balance these rewards to reach an acceptable result, which could take time due to trial-and-error.

% industria e una foarte profitabila, e o chestie mainstream
% un aspect important al jocurilor este AI-ul, care trebuie sa fie destul de bun incat sa ofere o provocare jucatorilor, dar sa nu fie unfair
% datorita faptului ca jocurile devin din ce in ce mai complexe, e din ce in ce mai greu sa dezvolti un AI 
% dezvoltarea cu RL are avantajul ca doar ii dai anumite reguli unui obiect si el invata in functie de ele, poate fi oarecum mai impredictable
% dezavantajul e ca trebuie sa vezi exact cum il pedepsesti sau recompensezi, plus ca dureaza sa il antrenezi
%
